\documentclass[12pt]{article}
\usepackage[utf8]{inputenc}
\usepackage{amsmath, amssymb}
\usepackage[a4paper,margin=1in]{geometry}
\usepackage{multicol}
\usepackage{fancyhdr}
\usepackage{tikz}
\pagestyle{fancy}
\fancyhf{}
\rhead{Newton's Laws and Kinematics}
\lhead{Physics Summary}
\rfoot{Page \thepage}

\title{Newton's Laws of Motion and Kinematic Equations}
\date{}
\begin{document}
\maketitle

\section*{Newton's Laws of Motion}

\subsection*{1. First Law (Law of Inertia)}
\textbf{Statement:} A body remains at rest or moves in a straight line with constant velocity unless acted upon by an external force.

\textbf{Mathematical Form:}
\[
\sum \vec{F} = 0 \quad \Rightarrow \quad \vec{v} = \text{constant}
\]

\subsection*{2. Second Law (Law of Acceleration)}
\textbf{Statement:} The acceleration of a body is directly proportional to the net external force acting on it and inversely proportional to its mass.

\textbf{Mathematical Form:}
\[
\sum \vec{F} = m \vec{a}
\]
where:
\begin{itemize}
    \item \( \vec{F} \) is the net force,
    \item \( m \) is the mass of the object,
    \item \( \vec{a} \) is the acceleration.
\end{itemize}

\subsection*{3. Third Law (Action-Reaction Law)}
\textbf{Statement:} For every action, there is an equal and opposite reaction.

\textbf{Mathematical Form:}
\[
\vec{F}_{12} = -\vec{F}_{21}
\]
This implies that if object 1 exerts a force on object 2, then object 2 exerts an equal and opposite force on object 1.

\section*{Equations of Motion (Constant Acceleration)}

For linear motion under constant acceleration, the kinematic equations are:

\begin{align*}
1. \quad v &= u + at \\
2. \quad s &= ut + \frac{1}{2}at^2 \\
3. \quad v^2 &= u^2 + 2as \\
4. \quad s &= \frac{(u + v)}{2} \cdot t
\end{align*}

Where:
\begin{itemize}
    \item \( u \): Initial velocity
    \item \( v \): Final velocity
    \item \( a \): Acceleration
    \item \( t \): Time
    \item \( s \): Displacement
\end{itemize}

\section*{Note}
These equations assume motion in a straight line with uniform acceleration and can be applied in any inertial reference frame.

\end{document}
