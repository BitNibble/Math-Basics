\chapter{Reference Manual}
%%%%%%%%%%%%%%%%%%%%%%%%%%%%%%%%%%%%%%%%%%%%%%%%%%%%%%%%%%
\section{Section}
$x(t) = V \times t$ \\
\(x^2 + y^2 = z^2\) \\
\[ x^n + y^n = z^n \] \\
\begin{math}...\end{math}
\noindent
\verb|\(...\)|
\texttt{\$...\$}
\begin{quote}
In physics, the mass-energy equivalence is stated 
by the equation \(E=mc^2\), discovered in 1905 by Albert Einstein.
\end{quote}

\begin{quote}
In physics, the mass-energy equivalence is stated 
by the equation $E=mc^2$, discovered in 1905 by Albert Einstein.
\end{quote}
\begin{quote}
In physics, the mass-energy equivalence is stated 
by the equation \begin{math}E=mc^2\end{math}, discovered in 1905 by Albert Einstein.
\end{quote}
\[...\]
\begin{displaymath}
...
\end{displaymath}
\begin{equation}
... \\
\sqrt{x^2+1}
\end{equation}
\newline
\[
\alpha \beta \gamma \rho \sigma \delta \epsilon
\times \otimes \oplus \cup \cap
< > \subset \supset \subseteq \supseteq
\int \oint \sum \prod
\]
\newline
\[ \int\limits_0^1 x^2 + y^2 \ dx \]
\[ \int_0^1 x^2 + y^2 \ dx \]
\[ a_1^2 + a_2^2 = a_3^2 \]
\[ x^{2 \alpha} - 1 = y_{ij} + y_{ij}  \]
\[ (a^n)^{r+s} = a^{nr+ns}  \]
\[ \sum_{i=1}^{\infty} \frac{1}{n^s} 
= \prod_p \frac{1}{1 - p^{-s}} \]
\[ \int\limits_0^1 x^2 + y^2 \ dx \]
\vspace{1cm}
\[ a_1^2 + a_2^2 = a_3^2 \]
\vspace{1cm}
\[ x^{2 \alpha} - 1 = y_{ij} + y_{ij}  \]
\vspace{1cm}
\[ (a^n)^{r+s} = a^{nr+ns} \]
\vspace{1cm}
\[ \sum_{i=1}^{\infty} \frac{1}{n^s} = \prod_p \frac{1}{1 - p^{-s}} \]
\vspace{1cm}
\[ \sqrt[4]{4ac} = \sqrt{4ac}\sqrt{4ac} \]
\[
a_{n_i}
\int_{i=1}^n
\sum_{i=1}^{\infty}
\prod_{i=1}^n
\cup_{i=1}^n
\cap_{i=1}^n
\oint_{i=1}^n
\coprod_{i=1}^n
\]
Let \( \mathcal{T} \) be a topological space, a basis is defined as
\[
 \mathcal{B} = \{B_{\alpha} \in \mathcal{T}\, |\,  U = \bigcup B_{\alpha} \forall U \in \mathcal{T} \}
\]
\begin{align*}
RQSZ \\
\mathcal{RQSZ} \\
\mathfrak{RQSZ} \\
\mathbb{RQSZ}
\end{align*}
\begin{align*}
3x^2 \in R \subset Q \\
\mathnormal{3x^2 \in R \subset Q} \\
\mathrm{3x^2 \in R \subset Q} \\
\mathit{3x^2 \in R \subset Q} \\
\mathbf{3x^2 \in R \subset Q} \\
\mathsf{3x^2 \in R \subset Q} \\
\mathtt{3x^2 \in R \subset Q}
\end{align*}
This is a simple example, {\tiny this will show different font sizes} and also \textsc{different font styles}
In this example the {\huge huge font size} is set and 
the {\footnotesize Foot note size also}. There's a fairly 
large set of font sizes
In this example, a command and a switch are used. 
\texttt{A command is used to change the style 
of a sentence}.
\sffamily
A switch changes the style from this point to 
the end of the document unless another switch is used
\renewcommand{\familydefault}{\sfdefault}
\renewcommand{\familydefault}{\rmdefault}
\tiny	F-tiny.png
\scriptsize	F-scriptsize.png
\footnotesize	F-footnotesize.png
\small	F-small.png
\normalsize	F-normalsize.png
\large	F-large.png
\Large	F-large2.png
\LARGE	F-large3.png
\huge	F-huge.png
\Huge	F-huge2.png
%%%%%%%%%%%%%%%%%%%%%%%%%%%%%%%%%%%%%%%%%%%%%%%%%%%%%%%%%%